\documentclass{article}
\usepackage[greek,english]{babel}
\usepackage{ucs}
\usepackage[utf8x]{inputenc}
\usepackage{hyperref}

\newcommand{\en}{\selectlanguage{english}}
\newcommand{\gr}{\selectlanguage{greek}}


\begin{document}
\title{Angelos Kamaris sdi1900070\\AI 2 - project 1}
\maketitle
\en
\gr
Καμάρης Άγγελος
\en
sdi1900070
\gr
\section{Πηγές}

\gr
Αξιοποίησα τον κώδικα του φροντιστηρίου κατά κύριο λόγο, χρησιμοποιώντας επίσης εντολές από το \en sklearn\gr . Αξιοποίησα τον οδηγό:\\ \en \href{https://www.kirenz.com/post/2021-12-11-text-mining-and-sentiment-analysis-with-nltk-and-pandas-in-python/text-mining-and-sentiment-analysis-with-nltk-and-pandas-in-python/}{https://www.kirenz.com/post/2021-12-11-text-mining-and-sentiment-analysis-with-nltk-and-pandas-in-python/text-mining-and-sentiment-analysis-with-nltk-and-pandas-in-python/}\\
\gr για \en NLTK preprocessing \gr των δεδομένων μου.\\
\gr
\section{Επεξήγηση Κώδικα}
Χρησιμοποίησα τις στήλες \en rating \gr και \en review \gr σαν Υ και Χ αντίστοιχα, όπου επεξεργάστηκα τα \en rating\gr , έτσι ώστε τα “κακά” (0-4) να έχουν την τιμή 0 και τα “καλά” (7-10) να έχουν την τιμή 1.\\\\
Επεξεργάστηκα τα δεδομένα από τα \en reviews \gr κάνοντας τα κεφαλαία μικρά, βγάζοντας τα σημεία στίξης, και τις λέξεις που είχαν λιγότερα από 2 γράμματα, βρίσκοντας τις πιο πολυχρησιμοποιημένες λέξεις και αφαιρώντας τις πιο σπάνιες, και τέλος κάνοντας \en lematize \gr και \en stematize \gr.\\\\
Χώρισα τα δεδομένα μου έτσι ώστε 30\% εξ αυτών να γίνονται validate και τα υπόλοιπα να χρησιμοποιούνται για το training, χρησιμοποίησα \en tf-idf\gr καθώς μου έφερνε καλύτερα αποτελέσματα, από τους άλλους που αναφέρθηκαν στο φροντιστήριο.\\\\
Χρησιμοποιώ \en logistic regression \gr για να κάνω \en train \gr τα δεδομένα μου και εμφανίζω τo την ευστοχία του με \en cross validation\gr . Αξιοποιώ το \en learning curve\gr  του φροντιστηρίου για να εκτυπώσω τις αποδόσεις του \en classifier \gr μου με τον αριθμό των δεδομένων, σύμφωνα με το \en F1-score \gr τους και το μέγεθος των δεδομένων. Τέλος εκτυπώνω τα: \en F1-Score, Recall, Precision\gr  για τα \en training\gr  και \en test\gr  που χώρισα πριν.\\\\
Αποφάσισα να βάλω \en max features \gr στο \en tf-idf=1000\gr , καθώς για κάτω από 500 δεν έχω εξίσου καλό \en accuracy \gr (κάτω από 0.86) και για πάνω από 1000 έχω \en overfitting\gr . Ο \en classifier \gr έχει \en max iter=2000\gr , μιας και δεν καθυστερεί με αυτό αλλά ούτε κάνει \en overfitting\gr . Τέλος έδωσα 30\% των δεδομένων μου στο\en  test\gr , ώστε να έχω αρκετά δεδομένα για \en train\gr .\\\\
Για την εισαγωγή ενός \en test\gr , αρκεί να εισάγετε τα δεδομένα στην μεταβλητή \en test df \gr , όπως το παράδειγμα στα σχόλια.\\\\
\section{Παρατηρήσεις}
Την μεγαλύτερη αλλαγή την παρατήρησα, στην χρήση \en preprocessing\gr , όταν το μοντέλο δεν έκανε \en overfit \gr ή \en underfit \gr καθώς παρατηρούνται αλλαγές της τάξης $0.2-0.8$, αναλόγως και τα υπόλοιπα δεδομένα.\\\\
Η κύρια αιτία \en overfitting \gr  ήταν ο μεγάλος αριθμός \en feature \gr στο \en vectorization \gr ενώ για \en underfitting \gr ήταν ο μικρός αριθμός δεδομένων εκπαίδευσης.\\\\
\section{Αποτελέσματα}
Τα αποτελέσματα που έχει το πρόγραμμά μου για τα δεδομένα που ανέφερα είναι:\\ \en
F1 Score (train): 0.88153321015877
F1 Score (validation): 0.8702659145850121
Recall Score (train): 0.8921779918864098
Recall Score (validation): 0.8823529411764706
Precision Score (train): 0.8711394442037507
Precision Score (validation): 0.8585055643879174



\end{document}
