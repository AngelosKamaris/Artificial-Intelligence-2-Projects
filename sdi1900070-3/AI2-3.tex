\documentclass{article}
\usepackage[greek,english]{babel}
\usepackage{ucs}
\usepackage[utf8x]{inputenc}
\usepackage{hyperref}

\newcommand{\en}{\selectlanguage{english}}
\newcommand{\gr}{\selectlanguage{greek}}


\begin{document}
\title{Angelos Kamaris sdi1900070\\AI 2 - project 2}
\maketitle
\en
\gr
Καμάρης Άγγελος
\en
sdi1900070
\gr
\section{Πηγές}

\gr
Αξιοποίησα τον κώδικα του φροντιστηρίου κατά κύριο λόγο, χρησιμοποιώντας επίσης εντολές από το \en sklearn, \gr καθώς επίσης αξιοποίησα σαν σκελετό την προηγούμενη εργασία μου, κρατόντας το \en preprocessing \gr των δεδομένων μου καθώς και τον χωρισμό των δεδομένων, την εισαγωγή τεστ και την εμφάνιση αποτελεσμάτων και το \en glove \gr.\\
\gr
\section{Επεξήγηση Κώδικα}
Χρησιμοποίησα τις στήλες \en rating \gr και \en review \gr σαν Υ και Χ αντίστοιχα, όπου επεξεργάστηκα τα \en rating\gr , έτσι ώστε τα “κακά” (0-4) να έχουν την τιμή 0 και τα “καλά” (7-10) να έχουν την τιμή 1.\\\\
Επεξεργάστηκα τα δεδομένα από τα \en reviews \gr κάνοντας τα κεφαλαία μικρά, αυτή την φορά κράτησα τα σημεία στίξης, έβγαλα τις λέξεις που είχαν λιγότερα από 2 γράμματα, βρίσκοντας τις πιο πολυχρησιμοποιημένες λέξεις και αφαιρώντας τις πιο σπάνιες, και τέλος έκανα \en lematize \gr , χωρίς  \en stematize \gr φυσικά για να αναγνωρίζονται οι λέξεις.\\\\
Χώρισα τα δεδομένα μου έτσι ώστε 30\% εξ αυτών να γίνονται \en validate \gr και τα υπόλοιπα να χρησιμοποιούνται για το \en training \gr και για την εισαγωγή των δεδομένων στο νευρονικό δίκτυο, τα πέρασα από το \en glove2word2vec \gr βάζοντας σαν \en input \gr το \en  glove.6B.300d.txt \gr καθώς με αυτό έχω την μεγαλύτερη ακρίβεια (από 76\% σε 84\%).\\\\
Δοκίμασα και \en LSTM \gr καθώς και \en GRU \gr και αν και γενικά τα αποτελέσματα ήταν αρκετά κοντά, προτίμησα το δεύτερο, παρόλου που θέλει περισσότερο χρονο. Του δίνω \en input size = 300, hidden size = 300, layers size = 1, output size = 2, clip=10 \gr και \en p=0.5, \gr καθώς αυτά μου έδωσαν τα καλύτερα αποτελέσματα ενώ για το \en LSTM \gr έχω : \en input size = 300, hidden size = 100, layers size = 1, output size = 2, clip=1000 \gr και \en p=0.6, \gr  . Θα παραθέσω μέσα στον φάκελο και τους φακέλους που έχω τα αποτελέσματα από τις συγκρίσεις μου, για τις οποίες θα μιλήσω σύντομα.\\\\
Συγκεκριμένα, φορτώνω τα δεδομένα που ανέφερα πριν στον \en classifier \gr μου, καλώ \en Linear \gr με \en output = hidden size*3 \gr καθώς δίνω 300 σαν \en hidden size \gr. καλώ επίσης \en clip, \gr για \en  gradient clipping, \en Dropout \gr για  \en dropout probability \gr καθώς επίσης χρησιμοποιώ σαν \en attention weight \gr το \en $nn.Parameter(torch.randn(hiddensize*2))$. \gr στο \en forward \gr καλώ το κελί που έχω ορίσει, εκτελώ  \en dropout,mechanism of attention, dropout \gr, με αυτή την σειρά (βρήκα τυχαία όταν είχα βάλει λάθος στο μοντέλο μου, ότι έτσι αποφεύγουμε περισσότερο το overfitting και το κράτησα), \en Linear \gr και βγάζω έναν πίνακα. Ενώ στο \en backwards \gr εκτελώ \en  clip grad norm  \gr.
Επειδή το αποτέλεσμα που θα μου δωθεί μετά το \en Linear \gr έχει έχει διάσταση τον αριθμό των \en inputs \gr επί 2, αφού τελειώσει το μοντέλο πρέπει να κρατήσουμε μόνο την μεγαλύτερη εκ των 2 τιμών.   

Για την εισαγωγή ενός \en test\gr , αρκεί να εισάγετε τα δεδομένα στην μεταβλητή \en test df \gr , όπως το παράδειγμα στα σχόλια.\\\\
\section{Παρατηρήσεις}
Την μεγαλύτερη αλλαγή την παρατήρησα, στην χρήση \en layers\gr , όταν το μοντέλο δεν έκανε \en overfit \gr ή \en underfit \gr καθώς παρατηρούνται αλλαγές της τάξης $0.02-0.04$, αναλόγως και τα υπόλοιπα δεδομένα, καθώς το μοντέλο μου δεν φαίνεται να επιθυμεί μεγάλο αριθμό, κάτι το οποίο θα το επιβράδυνε αν ήταν μεγαλύτερο του 1. \\\\
Το \en p \gr φαίνεται να επηρεάζει σε μεγάλο βαθμό το \en overfitting \gr, καθώς όσο μικρότερο είναι τόσο μεγαλύτερο \en overfitting \gr  θα κάνουμε αλλά για μεγαλύτερο, μικραίνει η ακρίβεια. Το \en hidden size \gr φαίνεται να επηρεάζουν επίσης πολύ τα αποτελέσματα, αναλόγως το κελί που χρησιμοποιούμε, καθώς το \en LSTM \gr προτιμά μικρό που το κάνει και ταχύτερο ενώ το \en GRU \gr προτιμά μεγάλο. Τέλος το \en clip \gr αν και μετά από ένα σημείο δεν υπάρχει μεγάλυ αλλαγή, είναι προτιμότερο να είναι σχετικά μεγάλος αριθμός.\\\\
Σε σχέση με τα μοντέλα από τις προηγούμενες εργασίες, αυτό το μοντέλο είναι πιο αποδοτικό, αν και θέλει πολύ περισσότερο χρόνο. Συγκεκριμένα, αποφεύγει περισσότερο το ovwrfitting, ενώ παράλληλα δίνει καλύτερα αποτελέσματα ως προς την ακρίβεια.\\\\
\section{Παραδοχές}
Έχω στείλει 2 αρχεία κώδικα, ένα για κάθε κελί, καθώς επίσης και 2 \en .txt \gr αρχεία με τα αποτελέσματα από διάφορες δοκιμές που έκανα. Μέσα υπάρχει επίσης και το μοντέλο του \en lstm \gr, καθώς και του \en gru \gr, τα οποία τα κατέβασα όπως είχαν δείξει στο φροντιστήριο.
\section{Αποτελέσματα}
Τα αποτελέσματα που έχει το πρόγραμμά μου για τα δεδομένα που ανέφερα είναι:\\ \en
LSTM\\
F1 Score (train): 0.8710447206123905
F1 Score (validation): 0.8549215406562054
Recall Score (train): 0.8953236493374108
Recall Score (validation): 0.8797709923664122
Precision Score (train): 0.8480477943395088
Precision Score (validation): 0.8314372918978913
GRU\\
F1 Score (train): 0.8789439793947199
F1 Score (validation): 0.8523263654753878
Recall Score (train): 0.8674928503336511
Recall Score (validation): 0.83980510851912
Precision Score (train): 0.8907014681892332
Precision Score (validation): 0.8652266504411318



\end{document}