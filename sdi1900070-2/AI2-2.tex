\documentclass{article}
\usepackage[greek,english]{babel}
\usepackage{ucs}
\usepackage[utf8x]{inputenc}
\usepackage{hyperref}

\newcommand{\en}{\selectlanguage{english}}
\newcommand{\gr}{\selectlanguage{greek}}


\begin{document}
\title{Angelos Kamaris sdi1900070\\AI 2 - project 2}
\maketitle
\en
\gr
Καμάρης Άγγελος
\en
sdi1900070
\gr
\section{Πηγές}

\gr
Αξιοποίησα τον κώδικα του φροντιστηρίου κατά κύριο λόγο, χρησιμοποιώντας επίσης εντολές από το \en sklearn\gr . Αξιοποίησα τον οδηγό:\\ \en \href{https://towardsdatascience.com/pytorch-tabular-binary-classification-a0368da5bb89}{https://towardsdatascience.com/pytorch-tabular-binary-classification-a0368da5bb89}\\
\gr για την δημιουργία του νευρωνικού δικτύου μου καθώς επίσης αξιοποίησα σαν σκελετό την προηγούμενη εργασία μου, κρατόντας το \en preprocessing \gr των δεδομένων μου καθώς και τον χωρισμό των δεδομένων, την εισαγωγή τεστ και την εμφάνιση αποτελεσμάτων.\\
\gr
\section{Επεξήγηση Κώδικα}
Χρησιμοποίησα τις στήλες \en rating \gr και \en review \gr σαν Υ και Χ αντίστοιχα, όπου επεξεργάστηκα τα \en rating\gr , έτσι ώστε τα “κακά” (0-4) να έχουν την τιμή 0 και τα “καλά” (7-10) να έχουν την τιμή 1.\\\\
Επεξεργάστηκα τα δεδομένα από τα \en reviews \gr κάνοντας τα κεφαλαία μικρά, αυτή την φορά κράτησα τα σημεία στίξης, έβγαλα τις λέξεις που είχαν λιγότερα από 2 γράμματα, βρίσκοντας τις πιο πολυχρησιμοποιημένες λέξεις και αφαιρώντας τις πιο σπάνιες, και τέλος έκανα \en lematize \gr , χωρίς  \en stematize \gr φυσικά για να αναγνωρίζονται οι λέξεις.\\\\
Χώρισα τα δεδομένα μου έτσι ώστε 30\% εξ αυτών να γίνονται \en validate \gr και τα υπόλοιπα να χρησιμοποιούνται για το \en training \gr και για την εισαγωγή των δεδομένων στο νευρονικό δίκτυο, τα πέρασα από το \en glove2word2vec \gr βάζοντας σαν \en input \gr το \en  glove.6B.300d.txt \gr καθώς με αυτό έχω την μεγαλύτερη ακρίβεια (από 76\% σε 84\%).\\\\
Χρησιμοποιώ \en Adadelta \gr για να κάνω \en optimization \gr καθώς μου έδωσε τα καλύτερα αποτελέσματα. Χρησιμοποιώ \en learning rate=0.001, batch size =20 , number of epochs=100\gr και το νευρωνικό μου δίκτυο έχει: \en h1=300, h2=150, h3=75\gr . Αξιοποιώ το \en learning curve\gr  της πρώτης εργασίας για να εκτυπώσω τις αποδόσεις του νευρωνικού δικτίου  μου με τον αριθμό των δεδομένων, σύμφωνα με το \en F1-score \gr τους και το μέγεθος των δεδομένων και εμφανίζω το \en losses plot\gr . Τέλος εκτυπώνω τα: \en F1-Score, Recall, Precision \gr για τα \en training \gr και \en test \gr  που χώρισα πριν καθώς και το \en ROC plot \gr.\\\\
Όλες οι αλλαγές στις οποίες αναφέρθηκα από πάνω αποσκοπούν στην εύρεση του καλύτερου αποτελέσματος. Ύστερα από δοκιμές είδα ότι χωρίς να κάνω \en overfitting \gr καταφέρνω να βρίσκω μεγαλύτερο \en validation \gr από \en train \gr, και καταλήγω με \en loss=2.63 \gr που είναι το χαμηλότερο που μπορώ να φτάσω σε λογικό χρόνο, χωρίς να χαλάω το \en recall \gr.
\\\
Για την εισαγωγή ενός \en test\gr , αρκεί να εισάγετε τα δεδομένα στην μεταβλητή \en test df \gr , όπως το παράδειγμα στα σχόλια.\\\\
\section{Παρατηρήσεις}
Την μεγαλύτερη αλλαγή την παρατήρησα, στην χρήση \en preprocessing\gr , όταν το μοντέλο δεν έκανε \en overfit \gr ή \en underfit \gr καθώς παρατηρούνται αλλαγές της τάξης $0.06-0.08$, αναλόγως και τα υπόλοιπα δεδομένα. Συγκεκριμένα τα σημεία στίξης φαίνεται να βοηθάνε το μοντέλο καθώς επίσης και όσο μεγαλύτερος ο πίνακας από το \en glove \gr τόσο μεγαλύτερο το \en accuracy.\gr\\\\
Το \en batch size \gr φαίνεται να επηρεάζει σε μεγάλο βαθμό το \en loss \gr, καθώς όσο μικρότερο είναι τόσο λιγότερο θα χάνουμε αλλά έτσι μειώνεται και το \en recall \gr. Τα \en layers \gr φαίνεται να επηρεάζουν επίσης πολύ τα αποτελέσματα, καθώς για πολύ μικρά, δεν έχουμε καλή ακρίβεια αλλά για πολύ μεγάλα το πρόγραμμα γίνεται αργό και μπορεί πάλι να καταλήξουμε με κακή ακρίβεια. Τέλος το \en learning rate \gr καθώς και ο \en optimizator \gr φαίνεται να έχουν την δυνατότητα να αυξήσουν την ακρίβεια καθώς και να μειώσουν το \en loss \gr ή να χαλάσουν την ακρίβεια.\\\\
\section{Αποτελέσματα}
Τα αποτελέσματα που έχει το πρόγραμμά μου για τα δεδομένα που ανέφερα είναι:\\ \en
F1 Score (train): 0.8423349699945445
F1 Score (validation): 0.8367087661609746
Recall Score (train): 0.8336403480912151
Recall Score (validation): 0.8274944567627495
Precision Score (train): 0.8512128680762745
Precision Score (validation): 0.846130592503023



\end{document}